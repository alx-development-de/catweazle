\documentclass[%
	a4paper,
	oneside,
	listof=numbered,
	parskip=half,
	headsepline=true,
	footsepline=false,
	0.7headlines,
	]{scrbook}

\usepackage{scrlayer-scrpage}
\usepackage[english]{babel}
\usepackage{csquotes}
\usepackage[fixlanguage]{babelbib}
\selectbiblanguage{english}
\usepackage{graphicx}
\usepackage{xcolor}
\usepackage{xltabular}
\usepackage{wrapfig}
\usepackage{adjustbox}
% listings package requires the scrhack to
% work properly and ommit a warning message
\usepackage{scrhack}
\usepackage{listings}
\usepackage{verbatim}
\usepackage{markdown}
\usepackage{tikz}
\usetikzlibrary{shapes,positioning}
\usepackage{typearea}
\areaset[1cm]{15cm}{23cm} % selbst setzen

% Style-Definition für die Code-Listings
\lstdefinestyle{mystyle}{
     basicstyle=\ttfamily\footnotesize,
     breakatwhitespace=false,
     breaklines=true,
     captionpos=t,
     frame=tlrb,
     keepspaces=true,
     numbersep=5pt,
     showspaces=false,
     showstringspaces=false,
     showtabs=false,
     tabsize=2,
     columns=fullflexible,
}
\lstset{style=mystyle}

\definecolor{gray}{rgb}{0.4,0.4,0.4}
\definecolor{darkblue}{rgb}{0.0,0.0,0.6}
\definecolor{cyan}{rgb}{0.0,0.6,0.6}

\lstdefinelanguage{bash}
{
  stringstyle=\color{black},
  identifierstyle=\color{darkblue},
  keywordstyle=\color{cyan},
  morekeywords={sudo,apt-key,apt,curl}% list your attributes here
}

\titlehead
{
	\adjustimage
	{
		width=\paperwidth,
		pagecenter,
		valign=B,
		}{cover-headline} % The headline image
	}

\pagestyle{scrheadings}

\begin{document}
% --------------------------------------------------------------
\title{CATweazle}
\subtitle{An environment for data handling in development and production of industrial cabinet building}
\subject{Industrial cabinet building}
\author{Dipl.-Ing. (FH) Alexander Thiel M.Sc.}

% Mit diesem Kommando kann das Datum festgelegt werden. Das ist sinnvoll, wenn dieser Bericht fertig ist und sich das Datum nicht immer, wenn er neu generiert wird wieder ändern soll.
%\date{30. Februar 2022}
%\publishers{...}

\maketitle
\tableofcontents
% --------------------------------------------------------------

\chapter{Introduction}\label{ch:introduction}

% Here we import the README.md file from theprojects base folder which should
% contain the introduction and a brief description
\markdownInput{../README.md}

\section{Install Portainer}\label{sec:install-portainer}

In this chapter, we’ll explain how to install Portainer on Ubuntu 22.04 with Docker.

Portainer is powerful, open-source toolset that allows you to easily build and manage
containers in Docker, Swarm, Kubernetes and Azure ACI. It works by hiding the complexity
that makes managing containers hard, behind an easy-to-use GUI .

\subsection{Prerequisites}\label{subsec:prerequisites}

\begin{enumerate}
\item Ubuntu 22.04 installed on a dedicated server or virtual machine.
\item A root user access or normal user with administrative privileges.
\item Add A record of your preferred domain like port.example.com
\end{enumerate}

First of all, keep server updated.
It's good practice to always keep your server up-to-date for security purpose.
On the Ubuntu system use the \verb|apt| command for the package management as shown below.

\begin{lstlisting}[%
	label={lst:apt-update},
	language=bash,
]
# sudo apt update -y
\end{lstlisting}

Now we install the required dependencies for Docker also using the \verb|apt| package manager.

\begin{lstlisting}[%
	label={lst:apt-install-docker-dependencies},
	language=bash,
]
# sudo apt install apt-transport-https ca-certificates curl gnupg-agent software-properties-common -y
\end{lstlisting}


Next, run the commands below to download and install Docker’s official GPG key.
The key is used to validate packages installed from Docker’s repository making sure they’re trusted.

\begin{lstlisting}[%
	label={lst:docker-install-gpg-keys1},
	language=bash,
]
# curl -fsSL https://download.docker.com/linux/ubuntu/gpg | sudo apt-key add -
# sudo apt-key fingerprint 0EBFCD88
\end{lstlisting}

Add the Docker Repository

\begin{lstlisting}[%
	label={lst:docker-install-gpg-keys2},
	language=bash,
]
# sudo add-apt-repository "deb [arch=amd64] https://download.docker.com/linux/ubuntu $(lsb_release -cs) stable"
\end{lstlisting}

The following command will download Docker and install it:

\begin{lstlisting}[%
	label={lst:xml_example_reference05},
	caption={Example of referencing objects},
	language=bash,
]
# sudo apt-get update -y
# sudo apt-get install docker-ce -y
\end{lstlisting}

Install Portainer on Ubuntu 22.04 with Docker

3. Create a container
We’ll show you two ways to deploy the container.

1.
If you want to use domain name to access Portainer, use following command to deploy the container:

\begin{lstlisting}[%
	label={lst:xml_example_reference06},
	language=bash,
]
# sudo docker run -restart always -d -name=portainer -v /var/run/docker.sock:/var/run/docker.sock -v /vol/portainer/data:/data -e VIRTUAL_HOST=port.example.com -e VIRTUAL_PORT=9000 portainer/portainer-ce -H unix:///var/run/docker.sock
\end{lstlisting}


Note:

-v /var/run/docker.sock:/var/run/docker.sock means mounting /var/run/docker.sock to the container so portainer can control the Docker.
-v /vol/portainer/data:/data means storing data of portainer on directory /vol/portainer/data.
port.example.com is your domain to access the portainer.

2.
If you want to access Portainer using server IP, use following command to deploy the container:

\begin{lstlisting}[%
	label={lst:docker-run-portainer},
	language=bash,
]
# sudo docker volume create portainer_data
# sudo docker run -d -p 8000:8000 -p 9000:9000 -name=portainer -restart=always -v /var/run/docker.sock:/var/run/docker.sock -v portainer_data:/data portainer/portainer-ce
\end{lstlisting}


4. Configure Reverse Proxy for Portainer (Optional if you will use domain name)
Caddyfile is a reverse proxy server.
It is necessary to secure the connection to prevent network hijacking.
Caddyfile can obtain and automatically maintains SSL certificate.

Create a Caddyfile.
Caddyfile is a document containing configs for your sites:

\begin{lstlisting}[%
	label={lst:xml_example_reference08},
	caption={Example of referencing objects},
	language=bash,
]
# sudo mkdir -p /vol/caddy/configs
# sudo vim /vol/caddy/configs/Caddyfile
\end{lstlisting}


Add following content:

\begin{lstlisting}[%
	label={lst:xml_example_reference09},
	caption={Example of referencing objects},
	language=bash,
]
port.example.com {
	tls youremail@example.com
	reverse_proxy portainer:8000
}
\end{lstlisting}


Replace: port.example.com with your domain name and youremail@example.com with your actual email id.

Save and exit.

Finally, create a Caddy container using following command:

\begin{lstlisting}[%
	label={lst:xml_example_reference},
	language=bash,
]
# sudo docker run -restart always -d -p 80:80 -p 443:443 -v "/vol/caddy/data:/data/caddy" -v "/vol/caddy/configs:/etc/caddy" -link portainer -name caddy caddy
\end{lstlisting}

Note:

-p 80:80 -p 443:443 means publish its 80 and 443 port to your host, so you can access it with those ports.
-v “/vol/caddy/data:/data/caddy” means mount caddy working directory to your host to persist data such as certificates.
-v “/vol/caddy/configs:/etc/caddy” means mount caddy configuration directory to your host to persist configurations.
–link portainer means link container caddy with portainer, so they can access with each other.
5.
Access Portainer
Navigate to your browser and access the Portainer by using either your domain or server IP and set admin password and finish the installment.



Install Portainer on Ubuntu 22.04 with Docker

That’s it.
The installation has been completed successfully.

% --------------------------------------------------------------
% Die Verzeichnisüberischten
\listoffigures
%\listoftables
\lstlistoflistings

%\bibliographystyle{babplain}
%\bibliography{ECAD-Export}

\end{document}
